\documentclass[13pt, t]{beamer}

% Presento style file
\usepackage{config/presento}

% custom command and packages
\input{config/custom-command}

\usebackgroundtemplate{\includegraphics[height=\paperwidth]{images/background.jpg}}

\title{\Large lingtypology: \\ R пакет для лингвистического картографирования}
\author[shortname]{\large Г. Мороз}
\institute[shortinst]{\large Лаборатория языковой конвергенции, НИУ ВШЭ}
\date{\large \begin{center} 18 декабря 2018 г. \bigskip \\ {\color{colorblue} Открытые лекции — «Городские данные»\\ Софт Культура и Инфокультура} \end{center}}

\begin{document}

% Title page
\begin{frame}[plain]
\maketitle
\end{frame}

\usebackgroundtemplate{\includegraphics[height=\paperheight]{images/background.jpg}}

\framecard[colorblue]{{\color{colorwhite} \huge \#тыжлингвист}}

\begin{frame}{\#тыжлингвист}
\begin{itemize}
\item  умеет читать на всех письменностях мира
\item знает все языки на свете
\item умеет распознавать каждый язык на слух \pause
\item может рассказать о происхождении каждого слова \pause
\item пишет без ошибок и знает все правила орфографии \pause
\item не знает математики и программирования \pause
\end{itemize}
\Large все вышеперечисленное, конечно, неправда
\end{frame}

\begin{frame}{Лингвистика}
\begin{itemize}
\item прескриптивная \pause
\item вся остальная
\begin{itemize}
\item исследования грамматики языка и языкового разнообразия
\item исследования распределения грамматических особенностей в языках мира
\item исследования когнитивных способностей человека и других животных, связанных с языком
\item исследования в области NLP и их приложения
\item исследования в области синтеза и распознования речи и языка
\item создание компьютерных инструментов для решения самых разных задач
\end{itemize}
\end{itemize}
\vfill
Еще бывает \textit{компьютерная лингвистика}, но это обобщенный термин, которым объединяют совсем несвязанные области:
\begin{itemize}
\item вспомогательные инструменты лингвистического исследования и документации
\item Computational linguistics
\item NLP
\end{itemize}
\end{frame}

\framecard[colorblue]{{\color{colorwhite} \huge лингвистические базы данных}}

\begin{frame}{Лингвистические базы данных}
\alert{\large Корпуса --- базы данных языкового материала}
\begin{itemize}
\item корпус литературных текстов
\item \href{http://ruscorpora.ru/}{\color{colorblue} Русский национальный корпус}
\item аудио и видео корпуса \pause
\begin{itemize}
\item настоящая речь, а не тексты
\item см., например, \href{http://rsl.nstu.ru}{\color{colorblue} корпус русского жестового языка}
\item см., например, \href{https://www.youtube.com/watch?v=OUwOvF7TqgA&feature=youtu.be&t=1m25s}{\color{colorblue} Уошо}
\end{itemize}
\end{itemize}
\vfill
\alert{\large Базы данных --- базы данных языковых структуру}
\begin{itemize}
\item \href{https://wals.info/}{The World Atlas of Language Structures (WALS)}
\item \href{https://ewave-atlas.org/}{World Atlas of Varieties of English (eWAVE)}
\item \href{https://glottolog.org/}{Glottolog} --- comprehensive reference information for the world's languages
\item ... \href{https://clld.org/datasets.html}{и другие}
\end{itemize}
\end{frame}

\end{document}